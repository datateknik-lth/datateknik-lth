\documentclass[a4paper]{article}
\usepackage[T1]{fontenc}
\usepackage[swedish]{babel}
\usepackage[utf8]{inputenc}
\usepackage{graphicx}
\usepackage{enumerate}
\title{Sammanfattning EDAF05}
\author{Meris Bahti \& Felix Mul}
\begin{document}
\maketitle
\newpage


\section{Deadlock Analysis}
Resource allocation graphs are used to determine if a program can deadlock.
For a program to end up in a deadlock there are a few requirements.
\begin{itemize}
  \item Mutual exclusion: at least one resource is held in a non-shareable mode.
  \item Hold and wait: there must exist a process that is holding at least one
        resource and simultaneously waiting for resources that are held by other
        processes.
  \item No preemption: resources cannot be preempted; the resource can only be 
        released voluntarily by the resource holding it.
  \item Circular wait: There must exist a set of processes waiting for each other
        in a circular structure. I.e: p1 waits for p2, p2 waits for p3, p3 waits
        for p1.
\end{itemize}

To draw a resource allocation graph from source code:
\begin{enumerate}
  \item Draw boxes for each resource.
  \item For each thread (i) and line (j), draw a bubble with $T_{ij}$. If a thread
        takes, then draw a line to the resource. For $T_{i(j+1)}$ draw a line from the
        resource to the thread.
  \item If $T_{ij}$ only emits or only absorbs arrows, you don't have to keep
        it in the graph.
  \item For resources that exist as multiple instances, draw dots inside the resource.
        If a cycle exists containing a multiple instance resource, then it may be a
        false cycle.
\end{enumerate}

Cycles in the graph indicate the possibility of deadlocks.
\begin{center}
  \includegraphics[scale=0.2]{res_alloc}
\end{center}

\section{}

\end{document}
