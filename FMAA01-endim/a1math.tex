\documentclass[11pt]{article}
\usepackage[T1]{fontenc}
\usepackage[swedish]{babel}
\usepackage[utf8x]{inputenc}


\usepackage{fancyvrb}
\usepackage{amssymb}
\usepackage{graphicx}
\usepackage{mathtools}

%\usepackage[section]{placeins}

\title{A1:an Repetition}
\author{Philip Larsson}
\date{\today}


\begin{document}
\maketitle


\section{Kapitel 1. Grundläggande begrepp och terminologi}
\subsection{Delmängd}
Om ändpunkterna ska räknas med används symbolerna [ ] och raka sträck $\vdash -- \dashv$. \\
Om ändpunkterna inte skall medräknas används ] [ och rundade sträck typ $\subset -- \supset$.
\subsection{Talsystem}
\begin{itemize}
\item{$\mathbb{Z}$ Heltal $\dots , -2, -1, 0, 1, 2, \dots$}
\item{$\mathbb{Q}$ Rationella tal. Bråktal/kvoter ex $\frac{2}{3}$. \\
Heltal är också rationella eftersom $5 = \frac{5}{1}$.}
\item{$\mathbb{R}$ Reella Tal. Alla tal som kan skrivas på en tallinje. Tal som $\pi$ och $\sqrt{2}$ samt även alla heltal och rationella tal.}
\end{itemize}

\section{Kapitel 2. Algebra}
\subsection{Räkneoperationer}
Konjugatregeln \\
$(a+b)(a-b)=a^2-b^2$ \\ \\
Kvadreringsregeln \\
$(a+b)^2 = a^2 + 2ab + b^2$ \\
$(a-b)^2 = a^2 - 2ab + b^2$
\subsection{Potenser}
$a^x * a^y = a^{x+y}$ \\
$(a^x)^y = a^{x*y}$ \\
$\frac{a^x}{a^y} = a^{x-y}$ \\
$a^0 = 1$ \\ \\
$a^{-x} = \frac{1}{a^x} $ \\
$(a * x)^x = a^x * b^x$ \\
$(\frac{a}{b}) = \frac{a^x}{b^x}$
\subsection{Polynom och rationella uttryck}
Exempel på polynom är \\
$p_1(x) = x^4 +2x -12x-5$ \\
$p_2(x) = (x-2)^2(x-4) + (x-2)(x+2)$ \\
Den korrekta matematiska förklaringen är att ett polynom kan skrivas på formen 
\begin{center}
$p(x) = a_n x^n + a_{n-1}x^{n-1} + \dots + a_1x + a_0 $
\end{center}
Ett tal kallas för nollställe till $p(x)$ om $p(\alpha) = 0$. Detta betyder att $\alpha$ är ett nollställe till $p(x)$. 
\subsection{Rationella uttryck}
Ett uttryck $(f(x))$ som kan skrivas på formen 
\begin{center}
$f(x) = \frac{p(x)}{q(x)} $
\end{center}
kallas för ett rationellt uttryck. \\
Exempel 
\begin{center}
$ \frac{x^3-x}{x^2 + x -2}$ $\frac{x^4-3x^2 -4}{x^3-7x+2}$ $\frac{x^4 - 2 x}{1}$
\end{center}
Observera att varje polynom är ett rationellt uttryck (med nämnare ett, som i sista exemplet ovan).  
\section{Kapitel 3. Ekvationer och olikheter}
En ekvation är ett matematiskt påstående som innehåller en likhet och vanligen en eller flera obekanta. \\
Exempel: 
\begin{center}
$x^2 -x - 2 = 0$ \\
$sinx = \frac{\sqrt{3}}{2} $
\end{center}
Att lösa en sådan ekvation är att bestämma alla värden på $x$ som uppfyller likheten. 
\subsection{Polynomekvationer}
Ekvationer av typen 
\begin{center}
$x^2 + px + q = 0$ 
\end{center}
kan lösas med hjälp av pq formeln 
\begin{center}
\fbox{$x=-\frac{p}{2} ± \sqrt{(\frac{p}{2})^2 -q}$}
\end{center}
\subsection{Olikheter}
Ledvis addition och subtraktion bevarar olikheter. Multiplikation eller division med ett positivt tal likaså. Däremot gäller det att ledvis multiplikation eller division med ett negativt tal ändrar riktning på olikhetstecknet. 

\section{Kapitel 4. Summor och talföljder}
Summatecknet $\sum$ \\ \\
$ \displaystyle\sum_{k=1}^{100} k = 1+2+3+ \ldots +99+100 $ \\
Det som står över summatecknet kallas för slutvärdet (100), och det som står under kallas för summationsindex och det är dess startvärde (k = 1). 
\subsection{Aritmetisk summa}
$\displaystyle\sum_{k=1}^{n} k = 1+2+3+\ldots+(n-1)+n = \frac{n(n+1)}{2}$
\subsection{Geometrisk summa}
$\displaystyle\sum_{k=0}^{n} x^k = 1+x+x^2+\ldots+x^{n-1}+x^n = \frac{x^{n+1}-1}{x-1}, x­1$
\subsection{Binomialsatsen}
K fakultet : \\
$k! = k \cdot (k-1) \cdot \ldots \cdot 2 \cdot 1 $ då $ k \geq 1 $ \\
Observera att $0! = 1$. \\ \\
Exempel på fakultet: \\
$3! = 3 \cdot 2 \cdot 1 = 6$ \\
$5! = 5 \cdot 4 \cdot 3 \cdot 2 \cdot 1 = 120$ \\ \\
Definition binomialkoefficienten. 
\begin{center}
${n\choose k} = \frac{n!}{k!(n-k)!}$ 
\end{center}
$n \choose k$ läses som ''n över k'' \\ \\
Binomialsatsen
\begin{center}
$(a+b)^n = a^n + {n \choose 1}a^{n-1}b^1 + {n \choose 2} a^{n-2}b^2 + \ldots + {n \choose n-1} a^1 b^{n-1} + b^n$
\end{center}
Kan också skrivas med hjälp av summatecken:
\begin{center}
$\displaystyle\sum_{k=0}^{n} {n \choose k} a^{n-k}b^k$
\end{center}

\section{Kapitel 5. Analytisk geometri}
Räta linjens ekvation: 
\begin{center}
\fbox{$y = kx + m$}
\end{center}
\emph{k} kallas för riktningskoefficienten och anger linjens riktning. \\
Konstanten \emph{m} anger vart den skär y-axeln.  \\ \\
Vi kan också ange en linje med enpunktsformeln. 
\begin{center}
\fbox{$y-y_1 = k(x-x_1)$}
\end{center}

\subsection{Parabel}
Parabler är en andragradskurva. 
\begin{figure}[h]
  \begin{center}
    \resizebox{!}{30mm}{\includegraphics{parabel}}
  \end{center}
  \label{bild3}
\end{figure} \\
Punkten där parabeln vänder kallas för vertex. 
\begin{center}
\begin{tabular}{l|r}
 \hline
  \multicolumn{2}{c}{Förflyttnings och omskalningsregler} \\
  \hline
  $x$ ersätts med $x-a$		&	förflyttning $a$ steg åt höger \\
  $x$ ersätts med $y-b$		&	förflyttning $a$ steg uppåt \\
  $x$ ersätts med $kx$		&	hoptryckning faktorn $k$ i x-led \\
  $y$ ersätts med $ky$		&	hoptryckning faktorn $k$ i y-led \\
  \end{tabular}
\end{center}
\subsection{Absolutbelopp och avstånd i planet}
Avståndet mellan $a$ och origo betecknar vi $|a|$. Avståndet kallas absolutbeloppet av $a$. \\
Observera att ett avstånd alltid är ett icke negativt tal. \\ \\
Ex. $|-7| = 7$ \\
Man skulle lite slarvigt kunna säga att absolutbeloppet plockar bort ett minustecken hos ett tal. \\
$|a-b|$ kommer att ge oss avståndet mellan $a$ och $b$.
\subsection{Cirkel ellips och hyperbel}
Andra andragradskurvor förutom parabel är cirkeln, ellipsen och hyperbeln. 

\subsubsection{Cirkel}
Cirkelns ekvation:
\begin{center}
\fbox{$(x-x_0)^2 + (y-y_0)^2 = r^2$}
\end{center}
\subsubsection{Ellips}
\begin{figure}[h]
  \begin{center}
    \resizebox{!}{30mm}{\includegraphics{ellips}}
  \end{center}
  \label{bild5}
\end{figure} 
En ellips med medelpunkt $(x_0,y_0)$ och halvaxlarna $a$ och $b$ består av mängden av alla punkter $(x,y)$ som uppfyller ekvationen : 
\begin{center}
\fbox{$\frac{(x-x_0)^2}{a^2} + \frac{(y-y_0)^2}{b^2} = 1$}
\end{center}
\subsubsection{Hyperbel}
\begin{figure}[h]
  \begin{center}
    \resizebox{!}{30mm}{\includegraphics{hyperbel}}
  \end{center}
  \label{bild3}
\end{figure} 
\begin{center}
\fbox{$\frac{(x-x_0)^2}{a^2} - \frac{(y-y_0)^2}{b^2} = 1$}
\end{center}
Med andra ord precis som en ellips fast med bytt tecken i vänsterledet. 
\section{Kapitel 7. Funktionsbegreppet}
Med en funktion menar vi en regel som till varje reellt tal (i någon given delmängd av $\mathbb{R}$) ordnar precis ett reellt tal. Ett exempel är den funktion $f$ som till vare reellt tal $x$ ordnar motsvarande kvadrerande tal $x^2$, dvs
\begin{center}
$f(x)=x^2$
\end{center}
En funktion består av två komponenter. En regel och en delmängd. \\
Ex:
\begin{center}
$f(x) = \sqrt{x-1}, x \geq 1$
\end{center}
Här har vi infört ett krav på variabeln, så funktionen är bara definierad för $x \geq 1$. Funktionen $f$ sägs då ha definitionsmängd $[1,\infty[$. Vi skriver då $D_f = [1,\infty[$
\\ \\
Avläsning på y-axeln ger oss alla värden som funktionen antar. Detta kallas funktionens värdemängd. 

\section{Kapitel 8. Elementära funktioner}
\subsection{Polynomfunktioner}
För en polynomfunktion av högst grad ett blir funktionskurvan en rät linje. Ett andragradspolynom ger upphov till en parabel. 
\begin{center}
\begin{figure}[h]
  \begin{center}
    \resizebox{!}{30mm}{\includegraphics{graphyx2}}
  \label{bild6}
    \resizebox{!}{30mm}{\includegraphics{graphx24x5}}
  \end{center}
  \label{bild7}
\end{figure} 
Graf $y=x+2$ och $y=x^2+4x+5$
\end{center}

\subsection{Potenser och exponentialfunktioner}
\subsubsection{Potensfunktion}
\begin{center}
$f(x) = x^a, x > 0 $
\end{center}
Exponenten är fix ($a$), men basen varierar i en potensfunktion. 
\subsubsection{Exponentialfunktion}
\begin{center}
$f(x) = a^x, x \in \mathbb{R}$
\end{center}
Om vi fixerar basen, men låter exponenten variera får vi en exponentialfunktion. \\
\\
Ett gemensamt drag för exponentialfunktioner är att oavsett bas så går grafen genom punkten $(0,1)$
\subsection{Logaritmfunktioner}
{\Large $^a\log x$} \\
Talet $a$ kallas logaritmens bas. 
\begin{center}
$b = ^a\log x \Leftrightarrow a^b = x$
\end{center}
Man kan lite slarvigt säga att a-logaritmen av $x$ är ''det tal som $a$ ska upphöjas med för att få $x$''. \\ 
Exempel:  \\
För att beräkna $^4\log16$ ställer vi oss frågan: Vilket tal ska 4 upphöjas med för att få 16? Svaret är 2, så slutsatsen är att $^4\log16=2$. \\
$\therefore ^4\log16 = ^4\log4^2 = 2$ \\ \\
Man kan se det som att $^4\log $ och 4:an ''tar ut varandra''. \\ \\
Tänk på att logaritmer endast är definierade för positiva tal. Exempelvis så saknar $^2\log(-5)$ mening eftersom $2^x = -5$ saknar lösning. \\ \\
Samtliga logaritm-grafer går igenom punkten $(1,0)$ eftersom
\begin{center}
$^a\log 1 = 0$
\end{center}
Vissa logaritmer har speciella beteckningar. 
\begin{itemize}
\item{$^e \log x = \ln x$, och kallas den naturliga logaritmen}
\item{$^{10}\log = \lg x$}
\end{itemize}
\subsubsection{Logaritm-regler} 
\begin{center}
$ logxy = logx + logy$ \\
$ log\frac{x}{y} = logx - logy $ \\
$ logx^k = k~logx $
\end{center}

\subsection{Trigonometriska funktioner}
\begin{center}
\begin{tabular}{p{40mm}p{40mm}}
$ cos \alpha = \frac{\text{närliggande katet}}{\text{hypotenusan}}$ & 
$ sin \alpha = \frac{\text{motstående katet}}{\text{hypotenusan}}$ \\ & \\
$ tan \alpha = \frac{\text{motstående katet}}{\text{närliggande katet}} $ &
$ cot \alpha = \frac{\text{närliggande katet}}{\text{motstående katet}} $ \\
\end{tabular}
\end{center}
\begin{figure}[h]
  \begin{center}
    \resizebox{!}{30mm}{\includegraphics{halvkvadrat.png}}
  \label{bild}
    \resizebox{!}{30mm}{\includegraphics{halvliksidigtriangel.png}}
  \end{center}
  \label{bild2}
\end{figure}

\subsubsection{Grader}
Med hjälp av ovanstående trianglar får vi nu ut följande värden: \\
\begin{center}
\begin{tabular}{lr}
	$ cos 45 ^\circ = sin 45 ^\circ = \frac{1}{\sqrt{2}} $	&	$ tan 45^\circ = cot 45^\circ = 1 $ \\ \\
	$ cos 60 ^\circ = sin 30 ^\circ = \frac{1}{2} $			&	$ tan 60^\circ = cot 30 ^\circ = \sqrt{3} $ \\ \\
	$ cos 30 ^\circ = sin 60 ^\circ = \frac{\sqrt{3}}{2} $	&	$ tan 30^\circ = cot 60 ^\circ = \frac{1}{\sqrt{3}} $ 

\end{tabular}	
\end{center}

\begin{figure}[h]
  \begin{center}
    \resizebox{!}{30mm}{\includegraphics{cossinus}}
  \end{center}
  \label{bild3}
\end{figure}
Cosinusvärdet läses på den vågräta axeln (x-axeln), och sinusvärdet på den lodräta (y-axeln). 
\begin{center}
\begin{figure}[h]
  \begin{center}
    \resizebox{!}{30mm}{\includegraphics{cosx}}
  \label{bild6}
    \resizebox{!}{30mm}{\includegraphics{sinx}}
  \end{center}
  \label{bild7}
\end{figure} 
$\cos x$ och $\sin x$
\end{center} 
Tan är definierat på följande sätt: 
\begin{center}
$tan x = \frac{sin x}{cos x}$
\end{center}


\subsubsection{Radianer}
Radianer är ett bättre sätt att ange vinklar. Vi utgår från enhetscirkeln och ett helt varv motsvaras av $2\pi$ radianer. Av detta följer det att: \\ 
\begin{center}
\begin{tabular}{lcr}
	$ \pi~rad = 180 ^\circ $	&	$ \frac{\pi}{2}~rad = 90^\circ $	&	$ \frac{\pi}{3}~rad= 60 ^\circ $ \\ \\
	$ \frac{\pi}{4}~rad= 45 ^\circ $ &	$ \frac{\pi}{6}~rad = 30^\circ $ 	&	$ \frac{\pi}{12}~rad= 15^\circ $ 
\end{tabular}
\end{center}
Detta leder till att: \\
\begin{center}
\begin{tabular}{lr}
	$ cos\frac{\pi}{4} = sin\frac{\pi}{4} = \frac{1}{\sqrt{2}}$	&	$ tan\frac{\pi}{4} = cot\frac{\pi}{4} = 1 $ \\ \\
	$ cos\frac{\pi}{3} = sin\frac{\pi}{6} = \frac{1}{2} $ &	$ tan \frac{\pi}{3} = cot\frac{\pi}{6} = \sqrt{3} $ \\Ê\\
	$ cos \frac{\pi}{6} = sin\frac{\pi}{3} = \frac{\sqrt{3}}{2}$	& 	$ tan\frac{\pi}{6} = cot\frac{\pi}{3} = \frac{1}{\sqrt{3}} $ 
\end{tabular}
\end{center}
\subsection{Trigonometriska funktioner}
Trigonometriska ettan: 
\begin{center}
$cos^2x + sin^2 x = 1$
\end{center}
\subsubsection{Sinus uttryckt i cosinus och vice versa}
$cos(-x) = cosx $ 	\\
$sin(-x) = -sinx $ 	\\
$cos(\frac{\pi}{2} -x) = sinx$ \\
$sin(\frac{\pi}{2} -x) = cosx$ \\
$cos^2x + sin^2x = 1$
\\ \\
Det går även att uttrycka kvadranterna $cos^2 x $ och $ sin^2 x$ såhär: 
\begin{center}
$cos^2x = \frac{1+ cos2x}{2}$ \\
$sin^2x = \frac{1-cos2x}{2}$
\end{center}
\subsubsection{Subtraktionsformler}
$ cos(x-y) = cosx~cosy + sinx~siny$ \\
$ sin(x-y) = sinx~cosy - cosx~siny$ 

\subsubsection{Additionsformel för cosinus och sinus}
$cos(x+y) = cosx~cosy-sinx~siny$ \\
$sin(x+y) = sinx~cosy + cosx~siny$ 

\subsubsection{Formlerna för dubbla vinkeln}
$ cos2x = cos^2x - sin^2x$ \\
$ sin2x = 2sinx~cosx $

\section{Area-, sinus- och cosinussatsen}
\emph{AREASATSEN} \\
Arean av en triangel är halva produkten av två sidors längder multiplicerat med sinus för mellanliggande vinkel. 
\begin{center}	
	$	T = \frac{bc}{2}sin\alpha $
\end{center}
\begin{figure}[h]
  \begin{center}
    \resizebox{!}{30mm}{\includegraphics{areasatsen}}
  \end{center}
  \label{bild3}
\end{figure}
\clearpage
\noindent\emph{SINUSSATSEN}
I en triangel med sidorna $ a, b, c$ och motstående vinklar $\alpha, ~\beta, ~\gamma$~ gäller \\
\begin{center}
$\frac{sin\alpha}{a} = \frac{sin\beta}{b} = \frac{sin\gamma}{c} $
\end{center}
\begin{figure}[h]
  \begin{center}
    \resizebox{!}{30mm}{\includegraphics{sinussatsen.png}}
  \end{center}
  \label{bild3}
\end{figure}
%\FloatBarrier
\emph{COSINUSSATSEN} \\
Om sidorna i en triangel är $a, ~b, ~c$ och den till sidan $a$ motstående vinkeln är $\alpha$, så gäller
\begin{center}
$ a^2 = b^2 + c^2 - 2bc~cos~\alpha $
\end{center}   
\begin{figure}[h]
  \begin{center}
    \resizebox{!}{30mm}{\includegraphics{cosinussatsen.png}}
  \end{center}
  \label{bild3}
\end{figure}


\subsection{Cirkelns ekvation}
Ekvation för en cirkel med given radie $r$ och given medelpunkt $(x_0,y_o)$. Låt $(x,y)$ vara en godtycklig punkt på cirkeln. Enligt definitionen av cirkel är avståndet mellan $(x_0,y_0)$ och $(x,y)$ konstant lika med $r$. 
\begin{center}
$ (x-x_0)^2 + (y-y_0)^2 = r^2 $
\end{center}





\subsection{Aritmetisk summa}
$ \sum\limits_{k=1}^{n} k=1+2+3+...+(n-1)+n = \frac{n(n+1)}{2} $

\subsection{Geometrisk summa}
$ \sum\limits_{k=0}^{n} x^k = 1+x+x^2+...+x^{n-1}+x^n= {x^{n+1} -1 \over x-1} $ \\
$ x \neq 1$

\subsection{Ellipsens ekvation}
$({x-x_0 \over a})^2 + ({y-y_o \over b})^2 = 1$ \\
Medelpunkt i $x_0 $ och $y_0$. \\
Talen a och b är ellipsens skärning med positiva x-axeln $(y=0)$ respektive y-axeln $(x=0)$. De kallas för ellipsens halvaxlar. 

\subsection{Definitionsmängd och värdemängd}
TO BE ADDED!

\subsection{Bevis av pythagoras sats}
Bilderna nedan visar en kvadrat vars sida är $a+b$ längdenheter lång. Genom att partionera kvadranten i två olika pussel kan man bevisa pythagoras sats.
\begin{center}
BILD
\end{center}
Eftersom de 4 trianglarna i båda pusslarna är lika stora så måste de båda streckande kvadranterna $(a^2 + b^2)$ ha lika stor area tillsammans som kvadraten i den högra $(c^2)$. \\
Alltså: $ a^2 + b^2 = c^2 $



\section{Summor och talföljder}
\subsection{Binomialkoefficient}
\begin{center}
$ {n \choose k} = { n!\over k!~(n-k)!} $ 
\end{center}
För heltal $ n, k \geq 0 $ med $ n \geq k$ definieras binomialkoefficienten $ {n \choose k} $
\subsection{Binomialsatsen}
\begin{center}
$(a+b)^n = \sum\limits_{k=0}^{n} {n \choose k} a^{n-k} b^k $
\end{center}
\section{Faktorsatsen}
Låt $f(x)$ vara ett polynom. \\
$f(a)=0 \Longleftrightarrow x-a $ delar $f(x)$ \\
\emph{Exempel} \\
\begin{center}
$ f(x) = x^3 -2x^2 - 11x +12 $ \\
$ f(1) = 1^2 - 2 -11 + 12 = 0$ \\
alltså delas f(x) av $ x=1 $ enligt faktorsatsen. 
\end{center}
\subsection{Korta versionen}
$f(x) = (x-a)q(x) + C \Longleftrightarrow C=f(a) $ \\
Alla $f(a) = 0 \Longleftrightarrow x-a $ delar $f(x)$ \\
($f(x)$ är ett polynom.)




\end{document}
